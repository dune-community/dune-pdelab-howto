%%%%%%%%%%%%%%%%%%%%%%%%%%%%%%%%%%%%%%%%%%%%%%%%%%%%%%%%%%%%
%%%%%%%%%%%%%%%%%%%%%%%%%%%%%%%%%%%%%%%%%%%%%%%%%%%%%%%%%%%%
%%%%%%%%%%%%%%%%%%%%%%%%%%%%%%%%%%%%%%%%%%%%%%%%%%%%%%%%%%%%
\section{Algebraic Formulation}
%%%%%%%%%%%%%%%%%%%%%%%%%%%%%%%%%%%%%%%%%%%%%%%%%%%%%%%%%%%%
%%%%%%%%%%%%%%%%%%%%%%%%%%%%%%%%%%%%%%%%%%%%%%%%%%%%%%%%%%%%
%%%%%%%%%%%%%%%%%%%%%%%%%%%%%%%%%%%%%%%%%%%%%%%%%%%%%%%%%%%%

\subsection{Solution of the Unconstrained Problem}

\paragraph{Unconstrained Problem in Original Basis}

\begin{frame}
\frametitle<presentation>{Unconstrained Problem in Original Basis}
We recall the unconstrained problem in weighted residual form:
\begin{equation}
u_h\in U_h\ : \qquad r_h(u_h,v) = 0 \qquad \forall
v\in V_h .
\end{equation}

Solving it in the original basis
reduces to the solution of a nonlinear algebraic problem:
\begin{equation}
\begin{split}
\mathbf{u}\in\mathbf{U} : \qquad
& r_h\left(\text{FE}_{\Phi_{U_h}}(\mathbf{u}),\psi_i\right) = 0, \quad
i\in\mathcal{I}_{V_h} \\
\Leftrightarrow \  & \mathcal{R}(\mathbf{u}) = \mathbf{0}
\end{split}
\end{equation}
where we introduced the nonlinear residual map $\mathcal{R} :
\mathbf{U} = \mathbb{K}^{\mathcal{I}_{V_h}} \to \mathbb{K}^{\mathcal{I}_{V_h}}$ which is defined as
\begin{equation}
\left(
\mathcal{R}(\mathbf{u})\right)_i =
r_h(\text{FE}_{U_h}(\mathbf{u}),\psi_i).
\end{equation}
$\Phi_{V_h} = \{\psi_i\,|\, i\in\mathcal{I}_{V_h}\}$ is the basis of $V_h$.
\end{frame}

\paragraph{Unconstrained Problem in Transformed Basis}

\begin{frame}
\frametitle<presentation>{Unconstrained Problem in Transformed Basis}
We may also solve the unconstrained problem
in the transformed basis for trial and test space:
\begin{equation}\label{Eq:TransformedUnconstrainedProblem}
\begin{split}
\mathbf{u}'\in\mathbf{U}' : \qquad
& r_h\left(\text{FE}_{\Phi'_{U_h}}(\mathbf{u}'),\psi_i'\right) = 0, \quad
i\in\mathcal{I}_{V_h}\\
\Leftrightarrow \  &
r_h\left(\text{FE}_{\Phi_{U_h}}(\mathbf{T}^T_{U_h}\mathbf{u}'),
\sum_{j\in\mathcal{I}_{V_h}}\left(\mathbf{T}_{V_h}\right)_{i,j}\psi_j\right) = 0, \quad
i\in\mathcal{I}_{V_h}\\
\Leftrightarrow \  &
\sum_{j\in\mathcal{I}_{V_h}} \left(\mathbf{T}_{V_h}\right)_{i,j}
r_h\left(\text{FE}_{\Phi_{U_h}}(\mathbf{T}^T_{U_h}\mathbf{u}'),
\psi_j\right) = 0, \quad
i\in\mathcal{I}_{V_h}\\
\Leftrightarrow \  &
\mathbf{T}_{V_h} \mathcal{R}\left(\mathbf{T}^T_{U_h}\mathbf{u}'\right)
= \mathbf{0} .
\end{split}
\end{equation}
Used linearity of residual form with respect to the second argument.

Requires simple matrix multiplication.
\end{frame}

\paragraph{Newton solver}

Use Newton's method to solve the algebraic problem.

\begin{frame}<article>
\frametitle<presentation>{Newton Solver}
Let a current iterate $\mathbf{u}_k'$ be given.

We seek an update $\mathbf{z}'_k$ such that $\mathbf{u}'_{k+1} = \mathbf{u}_k'
+ \mathbf{z}'_k$ and linearize:
\begin{equation*}
\mathbf{T}_{V_h}\mathcal{R}\left(\mathbf{T}^T_{U_h}\mathbf{u}'_{k+1}\right) \approx
\mathbf{T}_{V_h}\mathcal{R}\left(\mathbf{T}^T_{U_h}\mathbf{u}'_{k}\right) +
\mathbf{T}_{V_h}\nabla\mathcal{R}\left(\mathbf{T}^T_{U_h}\mathbf{u}'_{k}\right)
\mathbf{T}^T_{U_h} \mathbf{z}'_{k} = \mathbf{0} .
\end{equation*}

A linear system for the update is
\begin{equation}\label{eq:UnconstrainedUpdate}
\mathbf{T}_{V_h}\nabla\mathcal{R}\left(\mathbf{T}^T_{U_h}\mathbf{u}'_{k}\right)
\mathbf{T}^T_{U_h} \mathbf{z}'_{k} = -
\mathbf{T}_{V_h}\mathcal{R}\left(\mathbf{T}^T_{U_h}\mathbf{u}'_{k}\right) .
\end{equation}

$\nabla\mathcal{R}\left(\mathbf{u}_{k}\right)$ denotes the
Jacobian matrix of the map $\mathcal{R}$.

Multiplying the update equation with $\mathbf{T}^T_{U_h}$ from the left yields
\begin{equation}\label{eq:OriginalUpdate}
\mathbf{T}^T_{U_h}\mathbf{u}'_{k+1} = \mathbf{T}^T_{U_h}\mathbf{u}_k' +
\mathbf{T}^T_{U_h}\mathbf{z}'_k .
\end{equation}

Setting $\mathbf{u}_{k} := \mathbf{T}^T_{U_h}\mathbf{u}_k'$ allows us
now to write the Newton scheme with respect to the original basis.
\end{frame}


\begin{frame}
\frametitle<presentation>{Newton Solver (Contd.)}
\begin{Alg}[Newton's method for unconstrained problem]
Given the initial guess $\mathbf{u}_{0}$ iterate until convergence
\begin{enumerate}[i)]
\item Compute residual:
  $\mathbf{r}_k=\mathcal{R}\left(\mathbf{u}_{k}\right)$.
\item Transform residual: $\mathbf{r}_k' = \mathbf{T}_{V_h}
  \mathbf{r}_k$.
\item Solve update equation:
  $\mathbf{T}_{V_h}\nabla\mathcal{R}\left(\mathbf{u}_{k}\right)
\mathbf{T}^T_{U_h} \mathbf{z}'_{k} =  \mathbf{r}_k'$.
\item Transform update: $\mathbf{z}_{k} =
  \mathbf{T}^T_{U_h}\mathbf{z}'_k$.
\item Update: $\mathbf{u}_{k+1} = \mathbf{u}_k
- \mathbf{z}_k$. \hfill$\square$
\end{enumerate}
\end{Alg}

Two applications of the basis transformation, for the
residual and the update, are necessary in steps ii) and iv).

These transformations are cheap due to the structure of the
transformation.

In step (iii) the \textit{transformed}
Jacobian system is required.

\textit{All these transformations are done generically by PDELab}!
\end{frame}

\subsection{Solution of Constrained Problem}

We now turn to the constrained problem.

\paragraph{Reformulation in unconstrained space}

\begin{frame}<article>
\frametitle<presentation>{Reformulation in unconstrained space}
We recall the constrained problem in weighted residual form:
\begin{equation}\label{Eq:ConstrainedProblem}
u_h\in w_h + \tilde{U}_h\ : \qquad r_h(u_h,v) = 0 \quad \forall
v\in \tilde{V}_h .
\end{equation}

This problem can be reformulated in the unconstrained space by adding
a constrained equation:
\begin{Prp}
Let $P_h : U_h \to \tilde{U}_h$ be a projection (i.~e.~$P_h^2 = P_h$)
and assume that the affine shift is such that $P_h w_h = 0$. Then
\begin{equation}\label{Eq:ConstrainedProblemReformII}
u_h\in U_h\ : \qquad \left\{\begin{array}{ll}
r_h(u_h,v) = 0 \quad \forall v\in \tilde{V}_h\\
(I-P_h)u_h = w_h
\end{array}\right.
\end{equation}
is equivalent to \eqref{Eq:ConstrainedProblem}.

\mode<article>{
\textit{Proof}. Assume that \eqref{Eq:ConstrainedProblem} holds and
$P_h w_h = 0$. Since $u_h$ solves \eqref{Eq:ConstrainedProblem}
the first equation in \eqref{Eq:ConstrainedProblemReformII} clearly holds.
Moreover, we have $u_h = w_h + \tilde{u}_h$ with $\tilde{u}_h\in
\tilde{U}_h$ which allows us to write $u_h = w_h + P_h v_h$ for some
$v_h\in U_h$. When we can prove that $v_h=u_h$ we obtain the desired
$(I-P_h)u_h = w_h$. We now show that $v_h=u_h$:
Applying $P_h$ to both sides of the identity $u_h = w_h + P_h v_h$ yields
$P_h u_h = P_h w_h + P_h^2 v_h$. Using $P_h w_h = 0$ and $P_h^2 = P_h$
yields $P_h u_h = P_h v_h$. Thus we may identify $v_h$ and $u_h$ as
$v_h$ was arbitrary.\\
Assume now that \eqref{Eq:ConstrainedProblemReformII} holds. The first
equation of \eqref{Eq:ConstrainedProblemReformII} is the same as
\eqref{Eq:ConstrainedProblem}. From the
second equation we conclude $u_h = w_h + P_h u_h$, i.~e.~ $u_h\in w_h
+ \tilde{U}_h$.} \hfill$\square$
\end{Prp}
\end{frame}

\paragraph{Reformulated Problem in Coefficient Space}

We now seek to solve problem
\eqref{Eq:ConstrainedProblemReformII} in coefficient space.

\begin{frame}<article>
\frametitle<presentation>{Reformulated Problem in Coefficient Space}
The projection $P_h$ is taken from the follwing commutative diagram:
\begin{equation*}
\begin{CD}
U_h @>{P_h = \text{FE}_{\Phi'_{U_h}}
\mathbf{R}^T_{\tilde{\mathbf{U}}',\mathbf{U}'}
\mathbf{R}_{\tilde{\mathbf{U}}',\mathbf{U}'}
\text{FE}_{\Phi'_{U_h}}^{-1}}>> \tilde{U}_h\\
@A{\text{FE}_{\Phi'_{U_h}}}AA @AA{\text{FE}_{\Phi'_{U_h}}
\mathbf{R}^T_{\tilde{\mathbf{U}}'\mathbf{U}'}}A\\
\mathbf{U}' @>{\qquad\mathbf{R}_{\tilde{\mathbf{U}}',\mathbf{U}'}\qquad}>> \tilde{\mathbf{U}}'
\end{CD}
\end{equation*}

\begin{Prp}
Using this definition of $P_h$ the reformulated constrained
problem \eqref{Eq:ConstrainedProblemReformII} in coefficient space reads
\begin{equation}\label{Eq:ConstrainedProblemInCoefficientSpace}
\mathbf{u}'\in\mathbf{U}' : \qquad \left\{\begin{array}{rcl}
\mathbf{S}_{\tilde{\mathbf{V}}'}
\mathcal{R}\left(\mathbf{T}^T_{U_h}\mathbf{u}'\right)
& = & \mathbf{0}\\
\mathbf{R}_{\bar{\mathbf{U}}',\mathbf{U}'} \mathbf{u}' & = & \mathbf{w}'
\end{array}\right.
\end{equation}
with $\mathbf{S}_{\tilde{\mathbf{V}}'}=\mathbf{R}_{\tilde{\mathbf{V}}',\mathbf{V}'} +
\mathbf{T}_{\tilde{V}_h,\bar{V}_h}\mathbf{R}_{\bar{\mathbf{V}}',\mathbf{V}'}$
and $w_h =
FE_{\Phi_{U_h}'}(\mathbf{R}_{\bar{\mathbf{U}}',\mathbf{U}'}^T\mathbf{w}')$.
\hfill$\square$
\end{Prp}
\end{frame}

The idea in this formulation is that with respect to the transformed
basis the affine shift (for Dirichlet boundary conditions) can be
``encoded'' in the constrained degrees of freedom
$\mathbf{R}_{\bar{\mathbf{U}}',\mathbf{U}'} \mathbf{u}'$. This is
possible because the subspace $\tilde{U}_h$ is the image
of the unconstrained degrees of freedom
$\mathbf{R}_{\tilde{\mathbf{U}}',\mathbf{U}'} \mathbf{u}'$
and the decomposition is orthogonal
(i.~e.~$\mathbf{R}^T_{\bar{\mathbf{U}}',\mathbf{U}'}
\mathbf{R}_{\bar{\mathbf{U}}',\mathbf{U}'}$ and
$\mathbf{R}^T_{\tilde{\mathbf{U}}',\mathbf{U}'}
\mathbf{R}_{\tilde{\mathbf{U}}',\mathbf{U}'}$ are orthogonal
projections).


\begin{frame}<article>
\frametitle<presentation>{Proof of Proposition}
The second equation is seen as follows:
\begin{equation}\label{Eq:SideConditionCoefficient}
\begin{split}
&(I-P_h) u_h = w_h \\
\Leftrightarrow \quad &
\left(\text{FE}_{\Phi'_{U_h}}\text{FE}_{\Phi'_{U_h}}^{-1}
- \text{FE}_{\Phi'_{U_h}}
\mathbf{R}^T_{\tilde{\mathbf{U}}',\mathbf{U}'}
\mathbf{R}_{\tilde{\mathbf{U}}',\mathbf{U}'}
\text{FE}_{\Phi'_{U_h}}^{-1}\right)\text{FE}_{\Phi'_{U_h}}\mathbf{u}'
= \text{FE}_{\Phi'_{U_h}}
\mathbf{R}^T_{\bar{\mathbf{U}}',\mathbf{U}'} \mathbf{w}'\\
\Leftrightarrow \quad &
\left( \mathbf{I} - \mathbf{R}^T_{\tilde{\mathbf{U}}',\mathbf{U}'}
\mathbf{R}_{\tilde{\mathbf{U}}',\mathbf{U}'}\right) \mathbf{u}' =
\mathbf{R}^T_{\bar{\mathbf{U}}',\mathbf{U}'} \mathbf{w}' \\
\Leftrightarrow \quad &
\mathbf{R}^T_{\bar{\mathbf{U}}',\mathbf{U}'}
\mathbf{R}_{\bar{\mathbf{U}}',\mathbf{U}'} \mathbf{u}' =
\mathbf{R}^T_{\bar{\mathbf{U}}',\mathbf{U}'} \mathbf{w}'\\
\Leftrightarrow \quad &
\mathbf{R}_{\bar{\mathbf{U}}',\mathbf{U}'} \mathbf{u}' = \mathbf{w}' .
\end{split}
\end{equation}
\end{frame}

\begin{frame}<article>
\frametitle<presentation>{Proof of Proposition (Contd.)}
For the first equation in \eqref{Eq:ConstrainedProblemReformII}
we proceed as in \eqref{Eq:TransformedUnconstrainedProblem}
\begin{equation}\label{Eq:TransformedConstrainedProblem2}
\begin{split}
\mathbf{u}'\in\mathbf{U}' : \qquad
& r_h\left(\text{FE}_{\Phi'_{U_h}}(\mathbf{u}'),\psi_i'\right) = 0, \quad
i\in\tilde{\mathcal{I}}_{V_h}\\
\Leftrightarrow \  &
r_h\left(\text{FE}_{\Phi_{U_h}}(\mathbf{T}^T_{U_h}\mathbf{u}'),
\sum_{j\in\mathcal{I}_{V_h}}\left(\mathbf{T}_{V_h}\right)_{i,j}\psi_j\right) = 0, \quad
i\in\tilde{\mathcal{I}}_{V_h}\\
\Leftrightarrow \  &
\sum_{j\in\mathcal{I}_{V_h}} \left(\mathbf{T}_{V_h}\right)_{i,j}
r_h\left(\text{FE}_{\Phi_{U_h}}(\mathbf{T}^T_{U_h}\mathbf{u}'),
\psi_j\right) = 0, \quad
i\in\tilde{\mathcal{I}}_{V_h}\\
\Leftrightarrow \  &
\underbrace{\left(\mathbf{R}_{\tilde{\mathbf{V}}',\mathbf{V}'} +
\mathbf{T}_{\tilde{V}_h,\bar{V}_h}\mathbf{R}_{\bar{\mathbf{V}}',\mathbf{V}'}
\right)}_{\mathbf{S}_{\tilde{\mathbf{V}}'}}\mathcal{R}\left(\mathbf{T}^T_{U_h}\mathbf{u}'\right)=
\mathbf{S}_{\tilde{\mathbf{V}}'} \mathcal{R}\left(\mathbf{T}^T_{U_h}\mathbf{u}'\right)
= \mathbf{0} .
\end{split}
\end{equation}
Here we made use of the structure of the transformation
\eqref{Eq:StructureTransformation} in the final line.
\end{frame}


\paragraph{Newton's Method for Constrained Problem}

Newton's method applied to the constrained
problem \eqref{Eq:ConstrainedProblemInCoefficientSpace} is formulated
in the following alorithm.

\begin{frame}
\frametitle<presentation>{Newton's Method for Constrained Problem}
\begin{Alg}[Newton's method for constrained problem]\label{algo:ConstrainedNewton}
Let the initial guess $\mathbf{u}_{0}$ with
$\text{FE}_{\Phi_{U_h}}(\mathbf{u}_{0}) \in w_h + \tilde{U}_h$ be given.
Iterate until convergence
\begin{enumerate}[i)]
\item Compute residual:
  $\mathbf{r}_k=\mathcal{R}\left(\mathbf{u}_{k}\right)$.
\item Transform residual: $\mathbf{r}_k' = \mathbf{S}_{\tilde{\mathbf{V}}'}
  \mathbf{r}_k$.
\item Solve update equation:
\begin{equation*}
\left(\begin{array}{cc}
\mathbf{S}_{\tilde{\mathbf{V}}'} \nabla
\mathcal{R}\left(\mathbf{T}^T_{U_h}\mathbf{u}_{k}'\right)
\mathbf{S}^T_{\tilde{\mathbf{U}}'} & \mathbf{0}\\
\mathbf{0} & \mathbf{I}
\end{array}\right)
\left(\begin{array}{c}
\tilde{\mathbf{z}}_{k}'\\
\bar{\mathbf{z}}_{k}'
\end{array}\right) =
\left(\begin{array}{c}
\mathbf{r}_k'\\
\mathbf{0}
\end{array}\right)
\end{equation*}
and set $\mathbf{z}'_{k} = \left(\begin{smallmatrix}
\tilde{\mathbf{z}}_{k}'\\ \bar{\mathbf{z}}_{k}'
\end{smallmatrix}\right)$.
\item Transform update: $\mathbf{z}_{k} =
  \mathbf{T}^T_{U_h}\mathbf{z}'_k$. (This is where interpolation to
  hanging nodes is done).
\item Update: $\mathbf{u}_{k+1} = \mathbf{u}_k
- \mathbf{z}_k$. \hfill$\square$
\end{enumerate}
\end{Alg}
\end{frame}


\begin{frame}<article>
\frametitle<presentation>{Constraint Equation in Newton's Method}
Let $\mathbf{u}_{k}'$ be given.  Seek update $\mathbf{z}_{k}'$ s.t.
$\mathbf{u}_{k+1}' = \mathbf{u}_{k}' + \mathbf{z}_{k}'$.

Inserting $\mathbf{u}_{k+1}'$ into the second equation of
\eqref{Eq:ConstrainedProblemInCoefficientSpace} yields
\begin{equation}\label{Eq:SideCond}
\begin{split}
& \mathbf{R}_{\bar{\mathbf{U}}',\mathbf{U}'} \mathbf{u}_{k+1}'
= \mathbf{R}_{\bar{\mathbf{U}}',\mathbf{U}'} \mathbf{u}_{k}' +
\mathbf{R}_{\bar{\mathbf{U}}',\mathbf{U}'} \mathbf{z}_{k}'
 =  \bar{\mathbf{w}}'\\
\Leftrightarrow\qquad &
\mathbf{R}_{\bar{\mathbf{U}}',\mathbf{U}'} \mathbf{z}_{k}' =
\bar{\mathbf{w}}' - \mathbf{R}_{\bar{\mathbf{U}}',\mathbf{U}'}
\mathbf{u}_{k}' = \mathbf{0}\\
\Leftrightarrow\qquad &
\bar{\mathbf{z}}_{k}' = \mathbf{0}
\end{split}
\end{equation}
where we introduced $\mathbf{z}_{k}' =
\mathbf{R}^T_{\bar{\mathbf{U}}',\mathbf{U}'}
\bar{\mathbf{z}}_{k}'$ and used $\mathbf{R}_{\bar{\mathbf{U}}',\mathbf{U}'}
\mathbf{R}^T_{\bar{\mathbf{U}}',\mathbf{U}'}=\mathbf{I}$.
Note that the affine shift is not changed during the iteration:
\begin{equation}
\mathbf{R}_{\bar{\mathbf{U}}',\mathbf{U}'}\mathbf{u}_{k+1}' =
\mathbf{R}_{\bar{\mathbf{U}}',\mathbf{U}'} \mathbf{u}_{k}' +
\underbrace{\mathbf{R}_{\bar{\mathbf{U}}',\mathbf{U}'}
  \mathbf{z}_{k}'}_{= \mathbf{0}} =
\mathbf{R}_{\bar{\mathbf{U}}',\mathbf{U}'} \mathbf{u}_{k}' .
\end{equation}
Thus it is sufficient to satisfy the affine shift in the initial
guess $\mathbf{R}_{\bar{\mathbf{U}}',\mathbf{U}'} \mathbf{u}_{0}' =
\bar{\mathbf{w}}'$.
\end{frame}

\begin{frame}<article>
\frametitle<presentation>{Constraint Equation in Newton's Method (Contd.)}
Now insert  $\mathbf{u}_{k+1}'$ into the first equation of
\eqref{Eq:ConstrainedProblemInCoefficientSpace}:
\begin{equation}
\begin{split}
\mathbf{S}_{\tilde{\mathbf{V}}'}
&\mathcal{R}\left(\mathbf{T}^T_{U_h}\mathbf{u}_{k+1}'\right)
= \mathbf{S}_{\tilde{\mathbf{V}}'}
\mathcal{R}\left(\mathbf{T}^T_{U_h}\mathbf{u}_{k}' +
\mathbf{T}^T_{U_h}\mathbf{z}_{k}' \right)\\
&= \mathbf{S}_{\tilde{\mathbf{V}}'}
\mathcal{R}\left(\mathbf{T}^T_{U_h}\mathbf{u}_{k}' +
\mathbf{T}^T_{U_h} \left(\mathbf{R}^T_{\bar{\mathbf{U}}',\mathbf{U}'}
\underbrace{\mathbf{R}_{\bar{\mathbf{U}}',\mathbf{U}'}
  \mathbf{z}_{k}'}_{=\mathbf{0}, \text{ cf.\eqref{Eq:SideCond}}} +
\mathbf{R}^T_{\tilde{\mathbf{U}}',\mathbf{U}'}
\mathbf{R}_{\tilde{\mathbf{U}}',\mathbf{U}'} \mathbf{z}_{k}' \right)
\right)\\
&= \mathbf{S}_{\tilde{\mathbf{V}}'}
\mathcal{R}\left(\mathbf{T}^T_{U_h}\mathbf{u}_{k}' +
\mathbf{T}^T_{U_h} \mathbf{R}^T_{\tilde{\mathbf{U}}',\mathbf{U}'}
\mathbf{R}_{\tilde{\mathbf{U}}',\mathbf{U}'} \mathbf{z}_{k}' \right)\\
&= \mathbf{S}_{\tilde{\mathbf{V}}'}
\mathcal{R}\left(\mathbf{T}^T_{U_h}\mathbf{u}_{k}' +
\underbrace{\left(\mathbf{R}^T_{\tilde{\mathbf{U}}',\mathbf{U}'} +
\mathbf{R}^T_{\bar{\mathbf{U}}',\mathbf{U}'} \mathbf{T}^T_{\tilde{U}_h,\bar{U}_h}
\right)}_{=: \,\mathbf{S}^T_{\tilde{\mathbf{U}}'}}
\mathbf{R}_{\tilde{\mathbf{U}}',\mathbf{U}'} \mathbf{z}_{k}'
\right) \\
&=
\mathbf{S}_{\tilde{\mathbf{V}}'}
\mathcal{R}\left(\mathbf{T}^T_{U_h}\mathbf{u}_{k}' +
\mathbf{S}^T_{\tilde{\mathbf{U}}'}
\mathbf{R}_{\tilde{\mathbf{U}}',\mathbf{U}'}
\mathbf{R}^T_{\tilde{\mathbf{U}}',\mathbf{U}'} \tilde{\mathbf{z}}_{k}'
\right)
=
\mathbf{S}_{\tilde{\mathbf{V}}'}
\mathcal{R}\left(\mathbf{T}^T_{U_h}\mathbf{u}_{k}' +
\mathbf{S}^T_{\tilde{\mathbf{U}}'} \tilde{\mathbf{z}}_{k}'\right)
\end{split}
\end{equation}
where we introduced $\mathbf{z}_{k}' =
\mathbf{R}^T_{\tilde{\mathbf{U}}',\mathbf{U}'}
\tilde{\mathbf{z}}_{k}'$ and used $\mathbf{R}_{\tilde{\mathbf{U}}',\mathbf{U}'}
\mathbf{R}^T_{\tilde{\mathbf{U}}',\mathbf{U}'}=\mathbf{I}$.
\end{frame}


\begin{frame}<article>
\frametitle<presentation>{Constraint Equation in Newton's Method (Contd.)}
Linearization now gives
\begin{equation*}
\mathbf{S}_{\tilde{\mathbf{V}}'}
\mathcal{R}\left(\mathbf{T}^T_{U_h}\mathbf{u}_{k}' +
\mathbf{S}^T_{\tilde{\mathbf{U}}'} \tilde{\mathbf{z}}_{k}'\right)
\approx \mathbf{S}_{\tilde{\mathbf{V}}'}
\mathcal{R}\left(\mathbf{T}^T_{U_h}\mathbf{u}_{k}'\right)
+ \mathbf{S}_{\tilde{\mathbf{V}}'} \nabla
\mathcal{R}\left(\mathbf{T}^T_{U_h}\mathbf{u}_{k}'\right)
\mathbf{S}^T_{\tilde{\mathbf{U}}'} \tilde{\mathbf{z}}_{k}' = \mathbf{0}.
\end{equation*}
Thus the equation for the update reads
\begin{equation*}
\mathbf{S}_{\tilde{\mathbf{V}}'} \nabla
\mathcal{R}\left(\mathbf{T}^T_{U_h}\mathbf{u}_{k}'\right)
\mathbf{S}^T_{\tilde{\mathbf{U}}'} \tilde{\mathbf{z}}_{k}'
= - \mathbf{S}_{\tilde{\mathbf{V}}'}
\mathcal{R}\left(\mathbf{T}^T_{U_h}\mathbf{u}_{k}'\right) .
\end{equation*}
\end{frame}

\cleardoublepage
