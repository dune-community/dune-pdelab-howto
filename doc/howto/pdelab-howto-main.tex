%%%%%%%%%%%%%%%%%%%%%%%%%%%%%%%%%%%%%%%%%%%%%%%%%%%%%%%%%%%%%%%%%%%%%%%%%%%%%%%%
%%%%%%%%%%%%%%%%%%%%%%%%%%%%%%%%%%%%%%%%%%%%%%%%%%%%%%%%%%%%%%%%%%%%%%%%%%%%%%%%
%
% A general frame for lecture slides and lecture notes in one file
% using LaTeX beamer
%
%%%%%%%%%%%%%%%%%%%%%%%%%%%%%%%%%%%%%%%%%%%%%%%%%%%%%%%%%%%%%%%%%%%%%%%%%%%%%%%%
%%%%%%%%%%%%%%%%%%%%%%%%%%%%%%%%%%%%%%%%%%%%%%%%%%%%%%%%%%%%%%%%%%%%%%%%%%%%%%%%

% only for the article version
\mode<article> 
{
  \usepackage{amsfonts}
  \usepackage{amsthm}
  \usepackage{fullpage}
  \setlength{\parindent}{0pt}
  \setlength{\parskip}{1.3ex plus 0.5ex minus 0.2ex}
}

% only presentation 
\mode<presentation>
{
%  \usepackage{times}
  \usetheme{default}
%  \usetheme{Montpellier}
  \setbeamercovered{transparent}
  \setbeamertemplate{background canvas}[vertical shading][bottom=white!10,top=white!10]
  \setlength{\parindent}{0pt}
  \setlength{\parskip}{1.2ex plus 0.5ex minus 0.3ex}
  \setbeamertemplate{theorems}[numbered]
  \usefonttheme[onlysmall]{structurebold}
}

% all after
\usepackage{graphicx}
%\usepackage{multimedia}
\usepackage{psfrag}
\usepackage{listings}
\lstset{language=C++, basicstyle=\ttfamily,
  stringstyle=\ttfamily, commentstyle=\it, extendedchars=true}
\usepackage{curves}
\usepackage{epic}
\usepackage{calc}
\usepackage{fancybox}
\usepackage{xspace}
\usepackage{enumerate}
\usepackage{algorithmic}
\usepackage{algorithm}

\mode<article> 
{
\usepackage{hyperref}
}

%The theorems
\mode<article> 
{
\newtheoremstyle{mystyle}%
{3pt}%
{3pt}%
{}%
{}%
{\sffamily\bfseries}%
{.}%
{.5em}%
{}%
\theoremstyle{mystyle}
}
\mode<presentation> 
{
\theoremstyle{definition}
}
\newtheorem{Def}{Definiton}[section]
\newtheorem{Exm}[Def]{Example}
\newtheorem{Lem}[Def]{Lemma}
\newtheorem{Rem}[Def]{Remark}
\newtheorem{Rul}[Def]{Rule}
\newtheorem{Thm}[Def]{Theorem}
\newtheorem{Cor}[Def]{Corollary}
\newtheorem{Obs}[Def]{Observation}
\newtheorem{Ass}[Def]{Assumption}
\newtheorem{Alg}[Def]{Algorithm}

% Delete this, if you do not want the table of contents to pop up at
% the beginning of each subsection:
\AtBeginSection[]
{
  \begin{frame}<beamer>
    \frametitle{Contents}
\tableofcontents[currentsection,sectionstyle=show/hide,subsectionstyle=show/show/hide]
%    \tableofcontents[currentsection]
  \end{frame}
}

% Title definition
\mode<presentation>
{
  \title{\texttt{dune-pdelab} Howto}
  \author{Peter Bastian}
  \institute[IPVS]
  {
    Universit�t Heidelberg\\
    Interdisziplin�res Zentrum f�r Wissenschaftliches Rechnen\\
    Im Neuenheimer Feld 368, D-69120 Heidelberg\\
	email: \url{Peter.Bastian@iwr.uni-stuttgart.de}
  }
  \date{\today}
  \logo{\includegraphics[width=9mm]{./EPS/iwrlogo-klein.eps}}
}
\mode<article>
{
  \title{\texttt{dune-pdelab} Howto}
  \author{\textsc{Peter Bastian}\\
    Universit�t Heidelberg\\
    Interdisziplin�res Zentrum f�r Wissenschaftliches Rechnen\\
    Im Neuenheimer Feld 368, D-69120 Heidelberg\\
	email: \url{Peter.Bastian@iwr.uni-stuttgart.de}
  }
  \date{\today}
}

% logo nach oben
\mode<presentation>
{
% No navigation symbols and no lower logo
\setbeamertemplate{sidebar right}{}

% logo
\newsavebox{\logobox}
\sbox{\logobox}{%
    \hskip\paperwidth%
    \rlap{%
      % putting the logo should not change the vertical possition
      \vbox to 0pt{%
        \vskip-\paperheight%
        \vskip0.1cm%
        \llap{\insertlogo\hskip0.1cm}%
        % avoid overfull \vbox messages
        \vss%
      }%
    }%
}

\addtobeamertemplate{footline}{}{%
    \usebox{\logobox}%
}
}

% number equations within sections in article mode
\numberwithin{equation}{section}

% math symbols
\newcommand{\diffd}{\,d}

%%%%%%%%%%%%%%%%%%%%%%%%%%%%%%%%%%%%%%%%%%%%%%%%%%%%%%%%%%%%%%%%%%%%%%%%%%%%%%%%
%%%%%%%%%%%%%%%%%%%%%%%%%%%%%%%%%%%%%%%%%%%%%%%%%%%%%%%%%%%%%%%%%%%%%%%%%%%%%%%%
%
% now comes the individual stuff lecture by lecture
%
%%%%%%%%%%%%%%%%%%%%%%%%%%%%%%%%%%%%%%%%%%%%%%%%%%%%%%%%%%%%%%%%%%%%%%%%%%%%%%%%
%%%%%%%%%%%%%%%%%%%%%%%%%%%%%%%%%%%%%%%%%%%%%%%%%%%%%%%%%%%%%%%%%%%%%%%%%%%%%%%%

\begin{document}

\mode<presentation>
{
  \begin{frame}
    \titlepage
  \end{frame}
}
\mode<article>
{
\maketitle
}

\begin{abstract}
This article contains concepts for a general discretization module for
the ``Distributed Numerics Environment'' (DUNE). It should enable one
to build up a library of finite element methods in an easy and
extendable way that is closely related to the mathematical formulation
of finite element method. As the first necessary step an abstract
framework for formulating a large variety of finite element methods is
attempted. 
\end{abstract}

\mode<presentation>{
\begin{frame}<presentation>
\frametitle{Outline}
\tableofcontents[section,sectionstyle=show/show,subsectionstyle=hide/hide/hide] 
\end{frame}
}

\mode<article>
{
\tableofcontents
}

\mode<article>
{
\clearpage
}

%%%%%%%%%%%%%%%%%%%%%%%%%%%%%%%%%%%%%%%%%%%%%%%%%%%%%%%%%%%%
\section{Introduction}
%%%%%%%%%%%%%%%%%%%%%%%%%%%%%%%%%%%%%%%%%%%%%%%%%%%%%%%%%%%%

\subsection{What is missing in DUNE?}

\begin{frame}
\frametitle<presentation>{What is missing in DUNE?}
\begin{itemize}
\item FEM Framework. Define grid based discretization schemes for
arbitrary systems of PDEs in an easy way that allows one to exploit
all the features of DUNE.
\item Coupling of problems on different grids (overlapping and
non-overlapping, different dimension). Includes parallelization and
mapping of grids to parts of a parallel machine.
\item Robust solvers for coupled problems that are as general as
possible. 
\item Fast, robust and scalable grid manager for locally refined
unstructured grids.
\item Geometry module and CAD interface allowing definition of domains, coefficients and
boundary conditions independent of a  specific grid
manager. Integration in Salome framework?
\item HDF based persistent storage of meshes and functions. Scalable
to large data sets and read/write from many grid managers.
\item Workflow environment. High-level description of simulation
workflows. 
\end{itemize}
\cite{BrennerScott}
\end{frame}


% die weiterf�hrende Literatur wollen wir in der Pr�sentation nie zeigen
\mode<presentation>
{
\begin{frame}
\frametitle{Bibliography}
\bibliographystyle{alpha}
\bibliography{lit}
\end{frame}
}

% Literatur im Artikel jetzt hier
\mode<article>
{
\clearpage
\addcontentsline{toc}{section}{Literaturverzeichnis}
\bibliographystyle{alpha}
\bibliography{lit}
}


\end{document}
