%%%%%%%%%%%%%%%%%%%%%%%%%%%%%%%%%%%%%%%%%%%%%%%%%%%%%%%%%%%%
%%%%%%%%%%%%%%%%%%%%%%%%%%%%%%%%%%%%%%%%%%%%%%%%%%%%%%%%%%%%
%%%%%%%%%%%%%%%%%%%%%%%%%%%%%%%%%%%%%%%%%%%%%%%%%%%%%%%%%%%%
\section{Solving Stationary Problems}\label{Sec:EllipticProblems}
%%%%%%%%%%%%%%%%%%%%%%%%%%%%%%%%%%%%%%%%%%%%%%%%%%%%%%%%%%%%
%%%%%%%%%%%%%%%%%%%%%%%%%%%%%%%%%%%%%%%%%%%%%%%%%%%%%%%%%%%%
%%%%%%%%%%%%%%%%%%%%%%%%%%%%%%%%%%%%%%%%%%%%%%%%%%%%%%%%%%%%

\begin{frame}
\frametitle<presentation>{Abstraction}
\begin{itemize}
\item Code reuse requires abstraction.
\item Abstractions for function spaces.
\item Abstractions for PDE discretizations.
\end{itemize}
\end{frame}

\subsection{Unconstrained Elliptic Model Problem}

\begin{frame}
\frametitle{Problem and Weak Formulation}
Consider the following model problem:
\begin{subequations}
\begin{align*}
-\Delta u + a u &= f &&\text{in $\Omega\subset\mathbb{R}^d$ (open, connected)},\\
\nabla u \cdot n &= 0 &&\text{on $\partial\Omega$}.
\end{align*}
\end{subequations}
\medskip
Weak formulation. Set $U = H^1(\Omega)$.
\begin{equation*}
u\in U \quad : \quad \underbrace{\int_\Omega \nabla u \cdot \nabla v + 
a u v - f v \,dx}_{r(u,v)} = 0 \qquad \forall v\in U.
\end{equation*}
Has unique solution for $a(x)\geq a_0>0$.

We call $r(u,v)$ residual form.

Other boundary conditions are treated later.
\end{frame}

\begin{frame}
\frametitle{Conforming Finite Element Method}
Needs conforming triangulation $E_h^0 = \{e_o,\ldots,e_{N_h^0-1} \}$ of $\Omega$. 

Define the conforming finite element space
\begin{equation*}
U_h^k = \left\{  u\in C^0(\overline{\Omega}) \ : \ u|_{\Omega_e} \in P_{k_e} \forall e\in E_h^0\right\} \subset H^1(\Omega).
\end{equation*}
\begin{itemize}
\item $\Omega_e$: domain of element $e\in E_h^0$.
\item $P_k$: Polynomials of degree $k$.
\item $k_e$: Polynomial degree on element $e$.
\end{itemize}
Discrete problem then reads:
\begin{equation*}
u_h \in U_h^k \quad : \quad r(u_h,v) = 0 \qquad \forall v \in U_h^k.
\end{equation*}
\end{frame}

\begin{frame}
\frametitle{Affine Finite Element Spaces}
Construct functions in $U_h^k$ from local basis on reference elements:
\begin{equation*}
U_h^k\ni u_h(x) = \sum_{e\in E_h^0} \sum_{l=0}^{n(e)-1} (\mathbf{u})_{g(e,l)}
\, \hat{\phi}_{e,l}(\mu_e^{-1}(x)) \, \chi_e(x).
\end{equation*}
\begin{itemize}
\item $n(e)$: Number of basis functions on element $e\in E_h^0$.
\item $\hat\Omega_e$: Reference element of element $e\in E_h^0$.
\item $\mu_e : \hat\Omega_e \to \Omega_e$: Element transformation.
\item $\hat\phi_{e,l} : \hat\Omega_e \to \mathbb{R}$: Local basis function.
\item $\mathcal{I}_{U_h^k} = \{0,\ldots,N_{U_h^k}-1\}$: Global index set.
\item $g : E_h^0 \times \mathbb{N}_0 \to \mathcal{I}_{U_h^k}$: Local to global index map.
\item $\mathbf{u} \in \mathbf{U} = \mathbb{R}^{\mathcal{I}_{U_h^k}}$: Global vector of degrees of freedom.
\item $\chi_e$: Characteristic function of element $e$.
\end{itemize}
Note: We might have a different set of basis functions on each element.
\end{frame}

\begin{frame}
\frametitle{Global Basis; Finite Element Isomorphism}
For $j\in \mathcal{I}_{U_h^k}$ set $L(j) = \{ (e,l) \, : \, g(e,l) = j\}$ (all local degrees 
of freedom associated with global degree of freedom $j$.

Global basis:
\begin{equation*}
\Phi_{U_h^k} = \left\{ \phi_j(x) = \sum_{(e,l)\in L(j)} \hat{\phi}_{e,l}(\mu_e^{-1}(x)) \, \chi_e(x) 
\, : \, j \in \mathcal{I}_{U_h^k} \right\}.
\end{equation*}

Finite Element Isomorphism:
\begin{align*}
\text{FE}_{\Phi_{U_h^k}} : \mathbf{U} &\to U_h^k, &
\text{FE}_{\Phi_{U_h^k}}(\mathbf{u}) &= \sum_{j\in \mathcal{I}_{U_h^k}} (\mathbf{u})_j \phi_j.
\end{align*}
\end{frame}

\begin{frame}
\frametitle{Algebraic Problem}
Using the basis the discrete problem can be written equivalently as a (in general nonlinear) algebraic problem:
\begin{align*}
&&& u_h \in U_h^k \quad : \quad r(u_h,v) = 0 && \forall v \in U_h^k,\\
&\Leftrightarrow && \mathbf{u}\in\mathbf{U} \quad : \quad
r\left(\text{FE}_{\Phi_{U_h^k}}(\mathbf{u}),\phi_i\right) = 0 &&
i\in\mathcal{I}_{U_h^k}, \\
&\Leftrightarrow && \mathbf{u}\in\mathbf{U} \quad : \quad
\mathcal{R}(\mathbf{u}) = \mathbf{0}.
\end{align*}
where
\begin{align*}
\mathcal{R} &: \mathbf{U} \to \mathbf{U}, &
\left(\mathcal{R}(\mathbf{u}) \right)_i :=  r\left(\text{FE}_{\Phi_{U_h^k}}(\mathbf{u}),\phi_i\right) .
\end{align*}

For linear PDEs $\mathcal{R}$ is affine linear: $\mathcal{R}(\mathbf{u}) = \mathbf{A} \mathbf{u} - \mathbf{b}$.
\end{frame}

\begin{frame}
\frametitle{Residual Assembly}
\begin{equation*}
\begin{split}
&(\mathcal{R}(\mathbf{u}) )_i  = r\left(\text{FE}(\mathbf{u}),\phi_i\right)
= \sum_{e\in E_h^0} \int_{\Omega_e} \nabla \text{FE}(\mathbf{u}) \cdot \nabla\phi_i 
+ a \, \text{FE}_{\Phi_{U_h^k}}(\mathbf{u}) \phi_i - f \phi_i \,dx\\
&= \sum_{e\in E_h^0} \int_{\Omega_e} 
\left[ \sum_{l=0}^{n(e)-1} (\mathbf{u})_{g(e,l)} \nabla_x \hat\phi_{e,l}(\mu_e^{-1}(x))\right]
\cdot \nabla_x \underbrace{\hat\phi_{e,m}}_{g(e,m)=i}(\mu_e^{-1}(x))\\
& + a \, \left[ \sum_{l=0}^{n(e)-1} (\mathbf{u})_{g(e,l)} \hat\phi_{e,l}(\mu_e^{-1}(x)) \right] \hat\phi_{e,m}(\mu_e^{-1}(x))
- f \hat\phi_{e,m}(\mu_e^{-1}(x)) \, dx\\
&= \sum_{e\in E_h^0} \int_{\hat\Omega_e} 
\Biggl\{\left[ \sum_{l=0}^{n(e)-1} (\mathbf{u})_{g(e,l)} (\nabla \mu_e(\hat{x}))^{-T}\nabla_{\hat{x}} \hat\phi_{e,l}(\hat{x})\right]
\cdot (\nabla \mu_e(\hat{x}))^{-T} \nabla_{\hat{x}} \hat\phi_{e,m}(\hat{x})\\
& + a \, \left[ \sum_{l=0}^{n(e)-1} (\mathbf{u})_{g(e,l)} \hat\phi_{e,l}(\hat{x}) \right] \hat\phi_{e,m}(\hat{x})
- f \hat\phi_{e,m}(\hat{x}) \Biggr\} \text{det} \nabla\mu_e(\hat{x}) \, d\hat{x} .
\end{split}
\end{equation*}
\end{frame}

\begin{frame}
\frametitle{Local Operator}
Define restriction to local degrees of freedom
\begin{align*}
\mathbf{U}_e &= \mathbb{R}^{n(e)}, &
\mathbf{R}_e &: \mathbf{U} \to \mathbf{U}_e, &
\left(\mathbf{U}_e(\mathbf{u})\right)_l &= (\mathbf{u})_{g(e,l)} \quad 0\leq l < n(e).
\end{align*}
Define \textit{local operator} $\bm{\alpha}^{\text{vol}}_{h,e} : \mathbf{U}_e \to \mathbf{U}_e$ (user part):
\begin{equation*}
\begin{split}
&\bigl(\bm{\alpha}^{\text{vol}}_{h,e}({\color{cyan}\mathbf{u}})\bigr)_m  = \\
&\sum_{e\in E_h^0} \int_{\hat\Omega_e} 
\Biggl\{\left[ \sum_{l=0}^{n(e)-1} ({\color{cyan}\mathbf{u}})_{l} {\color{purple}
(\nabla \mu_e(\hat{x}))^{-T}} {\color{blue}\nabla_{\hat{x}} \hat\phi_{e,l}(\hat{x})} \right]
\cdot {\color{purple}(\nabla \mu_e(\hat{x}))^{-T}} {\color{blue}\nabla_{\hat{x}} \hat\phi_{e,m}(\hat{x})}\\
& + {\color{olive} a} \, \left[ \sum_{l=0}^{n(e)-1} ({\color{cyan}\mathbf{u}})_{l}
 {\color{blue}\hat\phi_{e,l}(\hat{x})} \right] {\color{blue}\hat\phi_{e,m}(\hat{x})}
- {\color{olive} f} {\color{blue}\hat\phi_{e,m}(\hat{x})} \Biggr\} {\color{purple}\text{det} \nabla\mu_e(\hat{x})} \, d\hat{x} .
\end{split}
\end{equation*}
Residual assembly is written generically:
\begin{equation*}
\mathcal{R}(\mathbf{u}) = \sum_{e\in E_h^0} \mathbf{R}_e^T \bm{\alpha}^{\text{vol}}_{h,e} (\mathbf{R}_e \mathbf{u})
\end{equation*}
\end{frame}


\begin{frame}
\frametitle{Solving the Algebraic System}
Use damped Newton method.

Given $\mathbf{u}^0\in\mathbf{U}$. Compute $\mathbf{r}^0 = \mathcal{R}(\mathbf{u}^0)$. Set $k=0$.

Iterate until convergence:
\begin{enumerate}
\item Assemble Jacobian System $\mathbf{A}^k = \nabla\mathcal{R}(\mathbf{u}^k)$.
\item Solve $\mathbf{A}^k \mathbf{z}^k = \mathbf{r}^k$ with some linear solver.
\item Update $\mathbf{u}^{k+1} = \mathbf{u}^{k} - \sigma^k \mathbf{z}^{k+1}$. $\sigma\in(0,1]$.
\item Compute new residual $\mathbf{r}^{k+1} = \mathcal{R}(\mathbf{u}^{k+1})$.
\item Set $k = k +1$.
\end{enumerate}

We need methods to compute $\mathcal{R}(\mathbf{u})$ and $\nabla\mathcal{R}(\mathbf{u})$.
\end{frame}

\begin{frame}
\frametitle{Jacobian}
The Jacobian matrix is defined as
\begin{equation*}
(\mathbf{A}^k)_{i,j} = (\nabla\mathcal{R}(\mathbf{u}^k))_{i,j} 
= \frac{\partial (\mathcal{R})_i}{\partial (\mathbf{u})_j}(\mathbf{u}^k)
= \sum_{e\in E_h^0} \frac{\partial (\bm{\alpha}_{h,e}^{\text{vol}})_m }{\partial (\mathbf{u})_l } (\mathbf{R}_e \mathbf{u}),
\end{equation*}
where $g(e,m)=l, g(e,l)=j$.

Again, the Jacobian can be computed from local contributions:
\begin{equation*}
\mathbf{A}^k = \sum_{e\in E_h^0} \mathbf{R}_e^T \nabla\bm{\alpha}_{h,e}^{\text{vol}}(\mathbf{R}_e \mathbf{u}) \, \mathbf{R}_e.
\end{equation*}

The local Jacobians can be
\begin{itemize}
\item programmed explicitly by the user, or
\item derived generically through numerical differentiation. This requires only coding
of the local residual contributions $\bm{\alpha}_{h,e}^{\text{vol}}$.
\end{itemize}
\end{frame}


\begin{frame}
\frametitle{The Linear Case}
is a special case of the nonlinear case \ldots
\begin{enumerate}
\item Given $\mathbf{u}^0\in\mathbf{U}$.
\item Compute $\mathbf{r} = \mathcal{R}(\mathbf{u}^0)$.
\item Assemble Jacobian System $\mathbf{A} = \nabla\mathcal{R}(\mathbf{u}^0)$.
\item Solve $\mathbf{A} \mathbf{z} = \mathbf{r}$ with some linear solver.
\item Update $\mathbf{u} = \mathbf{u}^{0} - \mathbf{z}$.
\end{enumerate}
\end{frame}



\begin{frame}<presentation>[fragile,allowframebreaks,allowdisplaybreaks]
\frametitle<presentation>{Unconstrained Elliptic Problem with $Q_1$}
\framesubtitle<presentation>{File \texttt{examples/examples01a\_Q1.hh}}
\lstinputlisting[basicstyle=\ttfamily\tiny,numbers=left, 
numberstyle=\tiny, numbersep=5pt]{../../examples/example01a_Q1.hh}
\end{frame}
\mode<article>{
\begin{Lst}[File examples/example01a\_Q1.hh] \mbox
\nopagebreak
\lstinputlisting[basicstyle=\ttfamily\scriptsize,numbers=left, 
numberstyle=\tiny, numbersep=5pt]{../../examples/example01a_Q1.hh}
\end{Lst}}


\begin{frame}<presentation>[fragile,allowframebreaks,allowdisplaybreaks]
\frametitle<presentation>{Local Operator for Unconstrained Elliptic Problem}
\framesubtitle<presentation>{File \texttt{examples/examples01a\_operator.hh}}
\lstinputlisting[basicstyle=\ttfamily\tiny,numbers=left, 
numberstyle=\tiny, numbersep=5pt]{../../examples/example01a_operator.hh}
\end{frame}
\mode<article>{
\begin{Lst}[File examples/example01a\_operator.hh] \mbox
\nopagebreak
\lstinputlisting[basicstyle=\ttfamily\scriptsize,numbers=left, 
numberstyle=\tiny, numbersep=5pt]{../../examples/example01a_operator.hh}
\end{Lst}}
